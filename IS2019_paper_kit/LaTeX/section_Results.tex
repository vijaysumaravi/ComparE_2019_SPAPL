% =====================================
\section{Experimental Results}
% ============

\subsection{Individual System Performance}
The effectiveness of eliminating the outliers in improving the robustness of the system was analyzed. A comparison between the use of all of the training data vs using the training data after removing outliers is shown in~Table~\ref{tab:results}. The results in terms of Spearman's correlation coefficient for entropy, VQual and MFCCs are shown using both statistics vector and i-vector to represent the utterances. 

% The system performance in terms of Spearman's correlation coefficient are shown in.  
%The baseline system used ComParE16 feature set. 

There is a consistent improvement in system performance by eliminating outliers from the training data. For example, when statistics vectors were used as utterance representation, outlier elimination improved the $\rho$ value from 0.158 to 0.192 for MFCCs and from 0.142 to 0.178 for VQual. Similarly, in the i-vector framework, outlier elimination improved the results for MFCCs from 0.248 to 0.252 and for VQual from 0.201 to 0.221.
Thus, the configuration with outlier elimination was selected for all individual systems.
%It is well known that outliers can cause a degradation in regression tasks~\cite{bollen1985regression} which may considerably reduce system performance. Improvements observed in prediction results when outlier elimination was performed was as expected.  

\begin{table}[b]
\centering

\caption{\label{tab:results}System performance in terms of Spearman's correlation for the development set using both statistics vector and i-vector representations. The results obtained after outlier elimination ($\rho_2$) are compared against the results using all of the training data ($\rho_1$). The best performing configuration for each individual feature set is boldfaced.} 

\begin{tabular*}{\linewidth}{l@{\extracolsep{\fill}}lcc}
\toprule
\textbf{Utt. representation} & \textbf{Feature set} & \bm{ $\rho_1$} &  \bm{ $\rho_2$}\\
\midrule
\midrule

\multirow{3}{*}{Statistics vector} 
% & ComParE16 & \textbf{0.251} \\
& entropy & 0.029 &   0.078  \\
& VQual & 0.142 &  0.179  \\
& MFCC & 0.158 & 0.192   \\
 \midrule
\multirow{3}{*}{i-vector}
& entropy & 0.126 &   \textbf{0.131 } \\
& VQual & 0.201 &  \textbf{0.221} \\
 & MFCC & 0.248&  \textbf{0.252} \\
 \bottomrule
\end{tabular*}%

\end{table}

The correlation between predicted degree and reference degree of sleepiness notably increased when the i-vector framework was used for utterance representation as compared to the statistics vector. This was observed for every acoustic feature. 
For MFCCs, for instance, the performance improved from 0.158 to 0.248, a relative improvement of approximately 56\%. 
When VQual features were used the improvement was around 40\% (from $\rho=0.142$ to 0.201). Therefore, the i-vector representation was chosen instead of statistics vector for each individual system.
%This improvement could partially be due to the compensation of speaker and phonetic variability. 


Although VQual and MFCC-based systems performed reasonably well ($\rho>0.2$) with the selected configuration, the entropy-based system did not perform as well. This could potentially be due to the varying effects of sleepiness on speech rates between read and spontaneous speech.
It was found that speakers' sleepiness was negatively correlated with speech rate for read speech, but an opposite pattern was observed for spontaneous speech \cite{Vogel2011}.
Unfortunately, no further analysis on the type of speech task could be made because such metadata was not available for the dataset used in this study.



\subsection{Fused System Performance}

Based on the results of the individual systems, VQual and MFCC-based systems using the i-vector representation with an outlier elimination prior were selected for fusion. As mentioned earlier, the baseline system ($\rho=0.251$) using ComParE16 and statistics vector was also used for fusion.
Table \ref{tab:results_score_fusion} presents the results for score fusion among the three individual systems. %Since outlier elimination provided better results, all experiments of score fusion were done with prior elimination of outliers in the training data.  

Score fusion with VQual improved performance for all cases.
For example, when VQual was fused with ComParE16, the $\rho$ score increased from 0.251 for ComParE16 alone to 0.283 for the fused system (a relative improvement of 12\%). %When MFFCs were fused with ComParE16, the improvement was approximately 18\%, from 0.251 to 0.296. 
Fusing with MFCCs also improved system performance ($\rho=0.265$) when compared to the MFCC-alone system ($\rho=0.252$). 
These results suggest that VQual provides complementary information about sleepiness to the MFCC and ComParE16 feature sets.

The best performance over all combinations of individual systems was when ComParE16, MFCCs and VQual features were fused altogether, resulting in the correlation score of 0.300. Compared to the baseline system, a relative improvement of 19.5\% was achieved. This system was used as the complete system.

%This improvement in performance when score-level fusion was applied can be explained with the complimentary nature of the feature sets. Results for the fusion for other features are not reported since no improvement in performance was observed. A noteworthy point is that across feature sets, fusing with VQual improved the results. 

\begin{table}[t]
\centering

\caption{\label{tab:results_score_fusion}System performance on the development dataset in terms of Spearman's correlation ($\rho$). Outlier elimination was used prior to sleepiness prediction. Statistics vector representation for ComParE16 and i-vector representation for MFCCs and VQual were used. The `+' sign indicates score level fusion. The best performing combination is boldfaced.}

\begin{tabular*}{\linewidth}{l@{\extracolsep{\fill}}l}
\toprule
\textbf{Feature sets} & \textbf{ $\rho$} \\
\midrule
\midrule

ComParE16+VQual & 0.283\\
\addlinespace
ComParE16+MFCCs & 0.296\\
\addlinespace
MFCCs+VQual & 0.265\\
\addlinespace
ComParE16+MFCCs+VQual &\textbf{0.300} \\
\addlinespace
 \bottomrule
\end{tabular*}%

\end{table}

\subsection{Performance on the Test Dataset}

Degree of sleepiness was predicted on the test dataset using the complete system on the development dataset. For the test phase, training and development datasets were combined to train the predictor.  %Outlier elimination was performed on this combined data followed by a score level fusion of individual predictors trained using ComParE16, MFCCs and VQual feature sets. 

The evaluation results of the complete system on development and test datasets are summarized in Table~\ref{tab:results_summary} in comparison to the baseline system. On the test data, the proposed system outperformed the baseline system with an improvement of 5.4\% ($\rho=0.314$ to 0.331). 


 \begin{table}[h]
\centering
\caption{\label{tab:results_summary}Complete system performance in terms of Spearman's correlation on the development and the test set. The baseline system performance using the ComParE16 feature set is also shown for comparison. Statistics vector representation for ComParE16 and i-vector representation for MFCCs  and  VQual  were used. The `+'  sign  indicates  score level fusion.}

\begin{tabular*}{\linewidth}{l@{\extracolsep{\fill}}cc}
\toprule
 \textbf{} & \textbf{Development} & \textbf{Test} \\
 \midrule
 \midrule
ComParE16 (baseline) & 0.251 & 0.314\\
\addlinespace
ComParE16+MFCCs+VQual & 0.300 & 0.331\\
\addlinespace
\bottomrule
\end{tabular*}
\end{table}