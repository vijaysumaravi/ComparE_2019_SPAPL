% =====================================
\section{Experimental Results}
% ============

\subsection{Single-Level System Performance}

System performance without outlier detection is first analyzed. A comparison of the effectiveness of the statistics vector and i-vector utterance representations is presented in Table \ref{tab:results_single_level_system}. The baseline system used ComParE16 feature set.

\begin{table}[h]
\centering

\caption{\label{tab:results_single_level_system}Correlation results for the development set without data processing. The best performing combination is boldfaced.}

\begin{tabular*}{\linewidth}{l@{\extracolsep{\fill}}lc}
\toprule
\textbf{Utt. representation} & \textbf{Feature set} & \textbf{ $\rho$} \\
\midrule
\midrule

\multirow{4}{*}{Statistics vector} 
% & ComParE16 & \textbf{0.251} \\
& entropy & 0.029 \\
& VQual & 0.142 \\
& MFCC & 0.158 \\
 \midrule
\multirow{3}{*}{i-vector}
& entropy & 0.126\\
& VQual & 0.201 \\
 & MFCC & \textbf{0.248} \\
 \bottomrule
\end{tabular*}%

\end{table}





Using the i-vector framework improved system performance substantially compared to using statistics vectors when the same acoustic features were used. 
For example, when MFCCs were used, the performance improved from 0.158 to 0.248, a relative improvement of approximately 56\%. When VQual features were used the, improvement was around 40\%, from a $\rho$ value of 0.142 to 0.201. This improvement could partially be due to the compensation of speaker and phonetic variability.


\begin{table}[h]
\centering

\caption{\label{tab:results_single_level_system}Correlation results for the development set with data processing. The best performing combination is boldfaced.}

\begin{tabular*}{\linewidth}{l@{\extracolsep{\fill}}lc}
\toprule
\textbf{Utt. representation} & \textbf{Feature set} & \textbf{ $\rho$} \\
\midrule
\midrule

\multirow{4}{*}{Statistics vector} 
% & ComParE16 &  \textbf{0.271}\\
& entropy & 0.078 \\
& VQual &  0.179\\
& MFCC &  0.192\\
 \midrule
\multirow{3}{*}{i-vector}
& entropy & 0.131\\
& VQual &  0.221\\
 & MFCC &  \textbf{0.252}\\
 \bottomrule
\end{tabular*}%

\end{table}

 




 The performance of a fused system was better than individual systems for both types of utterance representations. When statistics vectors were used, the score level fusion of MFCC and VQual improved the $\rho$ value from 0.158 for MFCC alone to 0.182 for the fusion system (15\% relative improvement). In case of ComParE16 features, fusing with VQual improved the $\rho$ value from 0.251 to 0.261. When i-vectors were used, the improvement in $\rho$ was from 0.248 for MFCC alone to 0.265 for the fused system (6\% relative improvement). This performance improvement with score level fusion can be explained with the complimentary nature of the feature sets. We do not report score fusion for other features since no improvement in performance was observed. We observe that across features, fusing with VQual improves results. 

 When between-frame changes in entropy is used in the single-level system the system performance is low. This might be potentially due the reason that 
 \Soo{You can refer to \cite{Vogel2011} that speech rate had opposite pattern between read and spontaneous speech}



\begin{table}[t]
\centering

\caption{\label{tab:outlier_results}Correlation results for the development set with data processing and score fusion. Statistics vector representation for ComParE feature set and i-vector representation for MFCCs and VQual were used. The `+' sign indicates score level fusion. The best performing combination is boldfaced.}

\begin{tabular*}{\linewidth}{l@{\extracolsep{\fill}}l}
\toprule
\textbf{Feature sets} & \textbf{ $\rho$} \\
\midrule
\midrule

ComParE16+VQual & 0.283\\
\addlinespace
ComParE16+MFCCs & 0.296\\
\addlinespace
ComParE16+MFCCs+VQual &\textbf{0.300} \\
\addlinespace
 \bottomrule
\end{tabular*}%

\end{table}

\subsection{Mulit-Level System Performance}
%We further build upon the single level system by implementing a multi-level system that predicts the sleepiness degree by first classifying the utterance and then doing a regression. 
The multi-level prediction method was implemented using only the i-vector utterance representation since the overall performance of the system using i-vectors is better than those using statistics vectors. 
Multi-level prediction results are shown in Table \ref{tab:results_multi_level_system}. 

Three different combinations of features were tested to implement the prediction. Each feature set -- entropy, Voice Quality and MFCC was used for classification (level 1) and the other two feature sets were used for regression, followed by a score fusion (level 2). The system performance was best when entropy is used for classification and MFCC and VQual were used for regression. 

Interestingly, although entropy did not perform better than other features for single-level prediction, it provided a performance gain when it was used for the classification followed by the regression using other feature sets. 

In brief, the best performing system was the one which used between-frame changes in entropy for classification followed by regression on VQual and MFCC with score level fusion. With a Spearman's correlation coefficient of 0.269,  our final system was able to achieve a relative improvement of 7.17\% over the baseline. 
The ComParE 2016 feature set provided with the baseline system had a $\rho$ of 0.251. 
% The confusion matrix for this is shown in Figure \ref{fig:}.  \Vijay{discussion on confusion matrix}

Speaker and phonetic variability poses a challenge to speaker state recognition~\cite{pisoni1989effects}. These may affect the acoustic variability arising due to the degree of sleepiness of a speaker thereby reducing the performance of sleepiness prediction systems. Bone~\etal~\cite{bone2011intoxicated} suggested the use speaker normalization to detect intoxicated speech.

\begin{table}[t]
\centering

\caption{\label{tab:outlier_results}Correlation results for the test set with data processing and score fusion. Statistics vector representation for ComParE feature set and i-vector representation for MFCCs and VQual were used. The `+' sign indicates score level fusion. the baseline system used ComParE16 feature set, as shown in the first row}

\begin{tabular*}{\linewidth}{l@{\extracolsep{\fill}}l}
\toprule
\textbf{Feature sets} & \textbf{ $\rho$} \\
\midrule
\midrule

ComParE16 & 0.314\\
\addlinespace
ComParE16+MFCCs+VQual & \\
\addlinespace
 \bottomrule
\end{tabular*}%

\end{table}