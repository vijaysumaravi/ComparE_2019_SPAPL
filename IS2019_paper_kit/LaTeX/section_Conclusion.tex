% =====================================
\section{Conclusion}
% =====================================
This study proposes the use of voice quality features (VQual) and between-frames changes in entropy for the prediction of sleepiness degree. VQual features are related to the perceived voice quality, and changes in entropy are an indirect measure of speech rate. An i-vector system was employed to reduce speaker and phonetic variability. To address the challenge of data imbalance, a multi-level prediction system was developed. A score-level fusion was done on complimentary features. 

In the single-level system, we observed a substantial gain in system performance when voice quality features were fused. In the case of multi-level system, the system performance was better when between-frame changes in entropy were used. Thus, the proposed acoustic feature set improved the performance of sleepiness prediction system. The i-vector system was an effective representation of the utterance. By implementing the i-vector representation for an utterance, we reduced variability in data arising due to factors extrinsic to speaker's degree of sleepiness. The resulted in a considerable improvement in system performance.  

Since the UBM was trained on an English database, it will be interesting in the future to develop a UBM on a German speech database. Developing an i-vector utterance representation for the ComParE16 features will be another important future study. 
