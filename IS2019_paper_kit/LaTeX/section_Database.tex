\section{Database}

\subsection{The SLEEP Corpus}

For the Continuous Sleepiness Sub-Challenge, a subset of the Duesseldorf Sleepy Language corpus collected from German speakers was used~\cite{schullerinterspeech}. Audio recordings were obtained with a sampling rate of 44.1~kHz and were down-sampled to 16~kHz. %\Soo{Audio should have recorded with 44.1 kHz sampling rate and 16-bit quantization, and have been downsampled later. Please double check. Or, just don't mention quantization}. 
The dataset consists of both read speech and spontaneous narrative speech. 

Speakers reported their sleepiness on the Karolinska Sleepiness Scale (KSS, \cite{aakerstedt1990subjective}) with a range of 1 (extremely alert) to 9 (very sleepy). Additionally, two observers assigned \textit{post hoc} KSS ratings.  
The self-assessed ratings by the speakers and the ratings from the observers were averaged to form the reference degree of sleepiness. 

% \begin{table}[t]
% \centering
% \caption{\label{tab:KSS}Distribution of training and development data among 9 degrees of sleepiness.\Soo{Do we need this table?}}

% \begin{tabular*}{\linewidth}{c@{\extracolsep{\fill}}cc}
% \toprule
%  & \multicolumn{2}{c}{\textbf{Number of utterances}} \\
% \cmidrule{2-3}
%  \textbf{KSS Rating} & \textbf{Training} & \textbf{Development} \\
%  \midrule
%  \midrule
% 1 & 121 & 61 \\
% 2 & 430 & 486 \\
% 3 & 934 & 978 \\
% 4 & 821 & 724 \\
% 5 & 816 & 843 \\
% 6 & 864 & 733 \\
% 7 & 848 & 664 \\
% 8 & 612 & 660 \\
% 9 & 118 & 179 \\
% \bottomrule
% \end{tabular*}
% \end{table}

\subsection{Databases for Training the i-vector Extractor}

Training a UBM and an i-vector extractor requires a database containing a large amount of recordings from multiple speakers. %But such a large database in German was not readily available for the given time frame. Instead, databases in English were utilized in this study.
%Because the amount of data in the SLEEP corpus was not sufficient to train the UBM and i-vector extractor,a separate set of databases was used for this purpose. in this study, 
The NIST SRE 04, 05, 06, and 08 databases~\cite{alvin2004nist,przybocki2006nist,martin2009nist} and the Switchboard II corpus phase 2 data~\cite{graff1999switchboard} were used. These databases provide more than 3,000 hours of speech samples in multiple languages from 3,408 female and 1,832 male speakers. 
The sampling rate of these recordings is 8~kHz. %The databases are chosen to provide a large number of speakers to train the i-vector system which compensates for speaker variability~\cite{dehak2011front}. 
